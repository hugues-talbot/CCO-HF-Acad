%%%%%%%%%%%%%%%%%%%%%%%%%%%%%%%%%%%%%%%%%
% Thin Sectioned Essay
% LaTeX Template
% Version 1.0 (3/8/13)
%
% This template has been downloaded from:
% http://www.LaTeXTemplates.com
%
% Original Author:
% Nicolas Diaz (nsdiaz@uc.cl) with extensive modifications by:
% Vel (vel@latextemplates.com)
%
% License:
% CC BY-NC-SA 3.0 (http://creativecommons.org/licenses/by-nc-sa/3.0/)
%
%%%%%%%%%%%%%%%%%%%%%%%%%%%%%%%%%%%%%%%%%

%----------------------------------------------------------------------------------------
%	PACKAGES AND OTHER DOCUMENT CONFIGURATIONS
%----------------------------------------------------------------------------------------

\documentclass[a4paper, 11pt]{article} % Font size (can be 10pt, 11pt or 12pt) and paper size (remove a4paper for US letter paper)
\usepackage[inner=2cm,outer=2cm]{geometry} %left=4cm,right=2cm would be equivalent

\usepackage[protrusion=true,expansion=true]{microtype} % Better typography
\usepackage{graphicx} % Required for including pictures
\usepackage{wrapfig} % Allows in-line images
\usepackage{subfigure}
\usepackage{amsmath}
\usepackage{mathpazo} % Use the Palatino font
\usepackage[T1]{fontenc} % Required for accented characters
\usepackage[toc,page]{appendix}
\newcommand{\sqlen}[1]{\ensuremath{(x - x_{#1})^2 + (y-y_{#1})^2}}
\newcommand{\rili}[1]{\ensuremath{\frac{r_{#1}^2}{l_{#1}}}}
\linespread{1.05} % Change line spacing here, Palatino benefits from a slight increase by default

\makeatletter
\renewcommand\@biblabel[1]{\textbf{#1.}} % Change the square brackets for each bibliography item from '[1]' to '1.'
\renewcommand{\@listI}{\itemsep=0pt} % Reduce the space between items in the itemize and enumerate environments and the bibliography

\renewcommand{\maketitle}{ % Customize the title - do not edit title and author name here, see the TITLE block below
\begin{flushright} % Right align
{\LARGE\@title} % Increase the font size of the title

\vspace{50pt} % Some vertical space between the title and author name

{\large\@author} % Author name
\\\@date % Date

\vspace{40pt} % Some vertical space between the author block and abstract
\end{flushright}
}

%----------------------------------------------------------------------------------------
%	TITLE
%----------------------------------------------------------------------------------------

\title{\textbf{Constrained Constructive Optimization implementation}\\ % Title
Reproducing Rudolph Karch paper} % Subtitle

\author{\textsc{Clara Jaquet} % Author
\\{\textit{Paris-Est University}}} % Institution

\date{\today} % Date

%----------------------------------------------------------------------------------------

\begin{document}

\maketitle % Print the title section

%----------------------------------------------------------------------------------------
%	ABSTRACT AND KEYWORDS
%----------------------------------------------------------------------------------------

%\renewcommand{\abstractname}{Summary} % Uncomment to change the name of the abstract to something else

\begin{abstract}
Implement CCO in 2D, then 3D based on Karch's work \cite{karch1999three}. In another step consider implementation for convex volume. 
Combination of CFD laws, geometry, and optimization. Encapsulated articles : requires summary in one paper.
\end{abstract}

\hspace*{3,6mm}\textit{Keywords:} constrained constructive optimization, implementation % Keywords

\vspace{30pt} % Some vertical space between the abstract and first section

\tableofcontents
%----------------------------------------------------------------------------------------
%	ESSAY BODY
%----------------------------------------------------------------------------------------

\section*{Introduction}

Constrained constructive optimization consists of growing a tree governed by minimizing a target function. 

In Karch's method the tree is constrained into a given convex perfusion volume, and the target function minimizes the total tree volume during growth. Segment are added one by one and fulfill both local optimization (single bifurcation scale) and global optimization (tree scale). The local optimization is based on Kamiya's work \cite{kamiya1972optimal}, whereas the global optimization has been implemented first by Schreiner in 2D \cite{schreiner1993computer}. 

\section{Global optimization}

Initialization of the method requires physiological parameters inputs, location of the root, and random location of first segment end.
Then Karch's protocol consists of a complete loop for each added segment. Picking a  random location under some constraints (geometry and physiology), used as a segment end candidate. Test its connection with neighbor segments, producing a bifurcation under topological, structural and functional constraints. In view of this new branch impact on the whole tree (induced by fluid mechanics laws), select the optimal one according to a target function (minimal volume of the tree). 


\subsection{Assumptions and boundary conditions}
The vascular tree is grown under assumptions and boundary conditions.
\subsection{Initialization step}  
Inputs: convex surface/volume, number of terminal segments,flow at root and conservation in the tree, pressure at root and terminal segments

\subsection{Loop to add new segment}

\subsubsection{Constrained new location}
pick random position

constrain it to volume and distance criteria

\subsubsection{Test connection}
if single bifurcation can be optimized

if no segment has degenerated to 0

if result doesn't overlap any other segments 

\subsubsection{Propagate impact on whole tree}
Propagate resistance change (balancing ratio)

\subsubsection{Measure target function}
Mesure total tree volume

\subsubsection{Select best connection between neighbors}
store result in cet: volume, betas, position

compare 

add the best one,then update flow, resistance, distance criteria and start again




\subsection{Example of results}
\subsubsection{2D}
\subsubsection{3D}
Output: images
decrease of radius, to maintain constant resistance


%------------------------------------------------

\section{Local optimization: single bifurcation scale}
Kamiya proposes a numerical solution to determine minimum volume bifurcation under restriction of physiological parameters, determinant pressure and flow, and locations at origin and terminals.

This method built in 2D assumes the flow to be laminar and vessels are composed of straight ducts lying on a plane. 

Note: Karch found that the optimum positions of the bifurcations in their 3D model trees were always found to lie in the plane defined by the endpoints of the respective three neighboring segments, which is consistent with the literature \cite{zamir1986branching}. 


\subsection*{Process: iterative nested loops}

Defining a starting position as the convex average of origin and terminal locations, weighted by respective flows.
\begin{equation}
(x,y) = (\frac{f_0x_0 + f_1x_1 + f_2x_2}{2f_0},\frac{f_0y_0 + f_1y_1 + f_2y_2 )}{2f_0})
\end{equation}
Calculate each segment length.
\begin{equation}
l_i^2 = (x - x_i)^2 + (y - y_i)^2
\label{length}
\end{equation}
Numerically calculate the new radii $r_0$, $r_1$,$r_2$. These ones are expected to satisfy both Hagen-Poiseuille's law and volume minimization.

When location of origin and two terminals segments, their pressure, and their flows are given, according to Hagen - Poiseuille's law:
\begin{align}
&\Delta P_1 = P_1 - P_0 = \kappa(\frac{f_0l_0}{r_0^4} + \frac{f_1l_1}{r_1^4}), \\
&\Delta P_2 = P_2 - P_0 = \kappa(\frac{f_0l_0}{r_0^4} + \frac{f_2l_2}{r_2^4})
\label{poiseuille}
\end{align}
  
According to  Kamiya, differentiating the tree volume with $x$, $y$ and $r_0$ and equating them to zero, one obtains:
\begin{equation}
\frac{r_0^6}{f_0} = \frac{r_1^6}{f_1} + \frac{r_2^6}{f_2}
\label{min volume}
\end{equation}
and 
\begin{equation}
x = \frac{x_0 r_0^2/l_0 + x_1 r_1^2/l_1 + x_2 r_2^2/l_2}{r_0^2/l_0 + r_1^2/l_1 + r_2^2 / l_2},
y = \frac{y_0 r_0^2/l_0 + y_1 r_1^2/l_1 + y_2 r_2^2/l_2}{r_0^2/l_0 + r_1^2/l_1 + r_2^2 / l_2}
\label{position}
\end{equation}
The details of the path to these equations is provided in appendix.

Using $R = r^2$ in \eqref{min volume}, we can express $R_0$ as:
\begin{equation}
R_0^3 = f_0(\frac{R_1^3}{f_1} + \frac{R_2^3}{f_2})
\end{equation} 
Substituting this inside \eqref{poiseuille}, one obtains the non linear system:
\begin{align}
\begin{cases}
&\frac{\Delta P_1}{\kappa}R_1^2 \left(f_0(\frac{R_1^3}{f_1} + \frac{R_2^3}{f_2}) \right)^\frac{2}{3} -f_0l_0R_1^2 - f_1l_1\left(f_0(\frac{R_1^3}{f_1} + \frac{R_2^3}{f_2}) \right)^\frac{2}{3} = 0 \\
&\frac{\Delta P_2}{\kappa}R_2^2 \left(f_0(\frac{R_1^3}{f_1} + \frac{R_2^3}{f_2}) \right)^\frac{2}{3} -f_0l_0R_2^2 - f_2l_2\left(f_0(\frac{R_1^3}{f_1} + \frac{R_2^3}{f_2}) \right)^\frac{2}{3} = 0
\end{cases}
\label{non linear system}
\end{align}

We are looking for the root satisfying these equations using a non linear solver. If the solution converges, we get the new radii, that are needed to calculate the new position of the branching point in \eqref{position}. This locations is a new input for the loop to iterate again (calculating length \eqref{length}, then new radii \eqref{non linear system}, new location \eqref{position} and so on). 
If this iterative loop converges and the bifurcation total volume decreases, the bifurcation is solved and provided with optimal radii and position.

Note: in CCO the pressure is not determined all along vessels (only at the root and terminal segments). In order to adapt to our situation, we use an estimated radius to calculate the pressure drops using \eqref{poiseuille}. At the first iteration the estimated radii are all equal to the segment's radius on which is connected the branch: $r_0 = r_1 = r_2 = r_{ori}$.
%At the first iteration the estimated radii correspond to the original segment radius (the one on which is added the connection): $r_0 = r_1 = r_{ori}$, and the radius of the last added segment to the tree (during the global growth): $r_2 = r_{n-1}$. 
Then, we will use the previously calculated radii to update the pressure drops at each iteration.

\subsection*{Example of results}
Example of results on different type of bifurcations. For these example we used a tolerance of 0.01 and maximum number of iteration of 100.\\

In the figure \ref{fig:generation} (a), for symetric flows and child locations: the optimal bifurcation point corresponds to convex average.

The figure \ref{fig:generation} (b) illustrates well Steiner solutions to optimal network \cite{bernot2009optimal}: it is more advantageous to transport flows together by delaying bifurcation. In the fluid mechanics context, this subadditivity follows from Poiseuille’s law, according to which the resistance of a tube increases when it gets thinner.

The figure \ref{fig:generation} (c) shows influence of blood demand on the bifurcation geometry: because flow is more important on the right child, the radius is bigger and the bifurcation is dragged toward this child. Also, we note the bifurcation is less delayed than for symmetrical flows.

The figure \ref{fig:generation} (d) shows influence of both destination and demand.

\begin{figure}[!h]
\centering
\subfigure[$Q_l = Q_r = \frac{1}{2} Q_p$]{\includegraphics[width=0.45\textwidth]{../../../SyntheticTreeGeneration/images/Test0/Superposition_FirstandLast.png}}
\subfigure[Random child locations and $Q_l = Q_r = \frac{1}{2} Q_p$]{\includegraphics[width=0.45\textwidth]{../../../SyntheticTreeGeneration/images/Test1_randomlocation/Superposition_Firstandlast.png} }\\
\subfigure[$Q_l = \frac{1}{4} Q_p$ and $Q_l = \frac{3}{4} Q_p$]{\includegraphics[width=0.45\textwidth]{../../../SyntheticTreeGeneration/images/Test2_randomflow/Superposition_first_and_last.png} }
\subfigure[Random child locations and $Q_l = \frac{1}{4} Q_p$, $Q_l = \frac{3}{4} Q_p$]{\includegraphics[width=0.45\textwidth]{../../../SyntheticTreeGeneration/images/Test4_randomlocationandflow/Superposition_Firstandlast.png} }

\caption{In blue the starting bifurcation (convex average position), in red the final bifurcation after Kamiya's algorithm convergence reached (tolerance = 0.01). Convergence was reached at 15th, 24th, 31st and 21st iteration respectively in (a),(b),(c),(d). $Q_p$ is the flow in parent branch, $Q_l$ and $Q_r$ are flows in left and right children.}
\label{fig:generation}
\end{figure}

\subsection*{Karch implementation}

Karch added lower and upper bounds to Kamiya's algorithm that ensures:
bifurcation position within the perfusion volume
the bifurcation not to degenerate to zero (by constraining segment length over segment diameter).


%------------------------------------------------

\section*{Conclusion}



\begin{appendices}
\section*{Equation \eqref{min volume}, from Kamiya \& Togawa 1972 equation  (6)}
Kamiya uses Murray definition at equation (7) from Physiological principle of minimum work\cite{murray1926physiological} that he calls the simplest requirement for efficiency in the circulation:
\begin{equation*}
f = k r^3
\end{equation*}
With k being a constant, so that the flow of blood past any section shall everywhere bear the same relation to the cube of the radius of the vessel at that point. Using it as:
\begin{equation*}
r_i^3 = \frac{r_i^6 k}{f_i}
\end{equation*}

and combining it this with the famous Murray's law,

\begin{equation*}
r_0^\gamma = r_1^\gamma + r_2^\gamma \text{ with } \gamma = 3
\end{equation*}

 where $r_0$ is the parent radius, $r_1$ and $r_2$ are the children radii, one obtains:
\begin{equation*}
\frac{k r_0^6}{f_0} = \frac{k r_1^6}{f_1} + \frac{k r_2^6}{f_2}
\end{equation*}
that can be simplified into equation \eqref{min volume}.


\section*{Equation \eqref{position}, from Kamiya \& Togawa 1972 equation  (7) }
We have
\begin{equation}
\label{eq:V}
V = \pi(r_0^2 l_0 + r_1^2 l_1 + r_2^2 l_2)
\end{equation}

and
\begin{align*}
l_0^2 &= \sqlen{0} \\
l_1^2 &= \sqlen{1} \\
l_2^2 &= \sqlen{2} 
\end{align*}

We rewrite \eqref{eq:V}

\begin{equation*}
V = \pi(r_0^2 \sqrt{\sqlen{0}} + r_1^2 \sqrt{\sqlen{1}} + r_2^2 \sqrt{\sqlen{2}})
\end{equation*}

We derive each term with respect to $x$.

\begin{equation*}
\frac{\partial}{\partial x} \sqrt{\sqlen{0}} = \frac{x-x_0}{\sqrt{\sqlen{0}}} = \frac{x-x_0}{l_0},
\end{equation*}
same for the $x_1$ and $x_2$ term, so we have

\begin{equation*}
\frac{\partial V}{\partial x} = \pi\left[ \frac{r_0^2(x-x_0)}{l_0} + \frac{r_1^2(x-x_1)}{l_1} + \frac{r_2^2(x-x_2)}{l_2}\right] = 0
\end{equation*}

Discarding the $\pi$ factor and separating the terms,

\begin{align*}
x\rili{0} + x\rili{1} + x\rili{2} &= x_0\rili{0} + x_1\rili{1} + x_2\rili{2} \\
x(\rili{0} + \rili{1} + \rili{2}) &= x_0\rili{0} + x_1\rili{1} + x_2\rili{2} 
\end{align*}

and so

\begin{equation*}
x = \frac{x_0\rili{0} + x_1\rili{1} + x_2\rili{2}}{\rili{0} + \rili{1} + \rili{2}}
\end{equation*}

This is one half of Eq.(7) in Kamiya \& Togawa. The other half is obtained by substituting $x$ with $y$ everywhere. This is
correct but not 100\% satisfying since the $l_i$ depend on $x$ and $y$.

\end{appendices}

%----------------------------------------------------------------------------------------
%	BIBLIOGRAPHY
%----------------------------------------------------------------------------------------

\bibliographystyle{unsrt}

\bibliography{sample}

%----------------------------------------------------------------------------------------

\end{document}